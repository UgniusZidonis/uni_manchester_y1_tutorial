\documentclass[a4paper]{article}
\usepackage[utf8]{inputenc}
\begin{document}
\begin{titlepage}
	\begin{center}
	\vspace*{1cm}
 
	\Huge{\textbf{Christmas reflection notes}}
 
	\vspace{1.5cm}
 
	\textbf{Ugnius Židonis}
 
 	\vfill
	\vspace{0.8cm}
 
	School of Computer Science\\
	The University of Manchester\\
	United Kingdom\\
	2019-02-01
	\end{center}
\end{titlepage}

\section*{Introduction}
In this essay I will be writing my reflections on the COMP10120 course unit. This includes: the best things about this course unit, the worst things, the most surprising things, the most important things that I have learned, what I have learned about my learning, how I work in a team, how I have developed since the start and what changes I plan on doing and how I will be achieving them.
\section*{The best things}
For me personally, since I am studying Computer Science, I like to learn everything digital, hence the best thing for me was ‘Phase 3 – World Wide What?’. In the beginning of this phase I merely remembered what I had learned before university, but after that I also learned some new web technologies and their usages. For example, I had never written any PHP before and I had always wondered why people hate it - now I can see the start of the reasons for the hate. Nonetheless, it is a simple and powerful tool for simple web applications or web pages. I also enjoyed the struggle of trying to remember how to write SQL. I had seen a fair amount of quality SQL code in my workplace and had the chance to write some of my own, unfortunately I had long forgotten the language.
\section*{The worst things}
I do not have good writing skills nor do I enjoy writing any kind of literature except for comedy, so the worst thing for me was and will be essays. I fully understand that that is one of the most popular and powerful ways of checking whether the student understands the material, however that does not change the fact that it is annoying and stressful for many students.
\section*{The most surprising things}
Mostly all universities have one problem in common – students acquire their desired degrees and leave without much practical knowledge. I was very surprised to see that this course unit tackles one of the most important skills of a developer – Google’ing. During Phase 3 we were instructed to find the required information on various web technologies, such as HTML and CSS, PHP, SQL, on our own using the power of online search engines. I thought that was a very modern approach for a university, since academia in general (at least in my country) is very slow to adapt to new methods of practice.
\section*{The most important things I have learned}
I believe the most important things were: teaching us how to do research individually, as I have stated my opinion on the importance of this skill in the previous paragraph; and getting us to discover the basics of web development, but not invest too much time in learning in-depth. Considering the amount of material there is online of the latter, it can be very challenging to determine how much is enough for, let’s say, a first year student. I think the amount of information was just enough to understand how and why those technologies are used while also not going in to too much detail.
\section*{What I have learned about my learning}
I have noticed that for short and simple tasks I prefer having strict requirements rather than being allowed to use my imagination. If a task is meant to be short, the amount of ways to tackle it should be narrowed down also, otherwise one person might spend more time thinking about how the problem could (not should) be solved, while another person has already finished the task flawlessly. Also, I tend to struggle my way through laboratories, trying to understand the material needed, without looking at the lecture slides or even watching a podcast of it. This is good and bad, in my opinion. The good part is that through this way of trying to understand the unit’s materials, I have a deeper understanding of how something works or doesn’t work because I have tested it myself. The bad part is that I might miss out on some key information that is either in the lectures slides or in the podcast. Overall,  I don’t think my learning habits have been toxic for me during the 1st semester.
\section*{How I work in a team}
This is a tricky question to answer because my team should answer this for me and not myself. Anyway, since I believe to have the most experience in the field (been working as a software engineer for over a year now), I never decline any challenging tasks (except for graphics designs, etc…). Which means I’m easy to work with since I never complain about “too much work”. However, I wouldn’t consider our previous tasks to be that difficult that they would require our best. Hence, I’m looking forward to seeing how we will function as a team while developing our first year project. I’m just worried that the project is too small for such a large team of 7 people. Still, we have been given a limited timeframe to finish our project, and there will be plenty of other work, considering we have other course units.
\section*{How have I developed}
I have a hard time answering this question. I seriously cannot think of anything other than my English speaking capabilities having improved and my calf muscles getting stronger since I don’t drive a car here.
\section*{What changes I need to do}
Considering the fact that there will likely be more work to do in the 2nd semester, I should get out of my home and try to work in public places, such as the universities’ learning commons or libraries. I have noticed that mixing up your free-time space with your work-place doesn’t end well. When you need to work, you can’t get around to doing actual work; when you need to relax, you can’t because you associate this place with work which is, of course, a stressful environment. Also, working in the learning commons will be quite convenient if we will have to work together with our team.
\section*{Last words}
This was a brief reflection of the COMP10120 course unit and my habits and struggles. Writing this essay wasn’t as annoying as I thought it would be. Maybe it’s because I’m writing it several hours before the deadline and I don’t have any time to complain about it.
\end{document}
