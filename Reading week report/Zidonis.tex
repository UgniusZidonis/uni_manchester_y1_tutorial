\documentclass[a4paper]{article}
\usepackage[utf8]{inputenc}
\begin{document}
\begin{titlepage}
	\begin{center}
	\vspace*{1cm}
 
	\Huge{\textbf{Common good}}
 
	\vspace{0.5cm}
	\Large An analysis of the ethical framework in \\The Killer Robot Case
 
	\vspace{1.5cm}
 
	\textbf{Ugnius Židonis}
 
 	\vfill
	\vspace{0.8cm}
 
	School of Computer Science\\
	The University of Manchester\\
	United Kingdom\\
	2018-11-01 
	\end{center}
\end{titlepage}

\section*{Introduction}
What exactly is "the common good"? The common good is a notion that originated over two thousand years ago in the writings of Plato, Aristotle, and Cicero. More recently, the ethicist John Rawls defined the common good as "certain general conditions that are equally to everyone's advantage". The Catholic religious tradition defines it as "the sum of those conditions of social life which allow social groups and their individual members relatively thorough and ready access to their own fulfilment." The common good, then, is based mostly of having the social systems, institutions, and environments on which we all depend work in a manner that benefits all people. Examples of the common good include an unpolluted and clean natural environment, an effective system of public safety, an accessible, reliable, and affordable public health care system, peace among the nations of the world, a just legal and political system, and a flourishing economic system. Because such systems and environments have such a powerful impact on the well-being of members of a society, it is no surprise that virtually every social problem in one way or another is linked to how well these systems and institutions are functioning.
\section*{Common good in The Killer Robot Case}
Considering the Killer Robot Case, there was a plethora of decisions that violated the ground rules of the common good framework. In effort to push production of a product, many people involved cut corners which lead to a major disaster – getting a person killed. As mentioned earlier, the common good focuses on benefiting every individual as equally as it is able, therefore we can understand why the death of a person is the most devastating consequence, since someone had to suffer because of others’ poor judgmenet and mistakes. In addition, some decisions that lead to this accident should have been thoroughly thought through before starting development of the robot. For example, at the beginning CEO Michael Waterson threatened the termination of the robotics department if the robot was not completed in time. This behaviour of the CEO is not clever nor nice, since it will most likely cause an unexpected accident if the company misses some details whilst rushing to deliver the product on time. This brings a huge risk to the users of the robot. In effort to reduce risk the CEO should have rather consulted experts about the time needed to have a fully functional and safe robot. Bear in mind that this was practically the first decision made on the project, which shows the motivation of the manufacturer. Another example would be Sam Reynolds’ decision of using the waterfall method over the agile method. Considering that others argued against Reynolds’ decision, he was being too dictatorial and making poor choices that eventually lead to failure or a more difficult approach to the final product. One must overcome their insecurities (in this case, a need to feel power) in order to have a balanced out system for work. So, this decision not only did not benefit others involved, it was made by failing to listen to a more experienced individual. Although these fundamental decisions had a great impact on increasing risk of failure or injury, the most irresponsible action taken was failing to train personel how to use the robot and how to deal with emergencies. If the operator that was killed had been instructed beforehand what actions to take in case of an emergency or how to even avoid such situations, most likely he would have been left uninjured. Nontheless, the evidence presented in The Killer Robot Case shows that a faulty UI design lead to the operator being unable to stop the machine from continuing functioning. This greedy need of finishing the software quicker rather than spending more time on a good quality interface is a prime example of what not to do using the common good guideline.  All of these decisions together lead to disaster because they did not follow ground rules that they should have agreed upon before starting the project. We can see that quite a few problems could have been avoided if the common good rules were applied but how would other problems be solved using another ethical framework?
\section*{Other frameworks in The Killer Robot Case}
There are various ways of viewing different scenarios, hence we do not have one particular framework, rather we have a list to choose from for individual cases. For example, if the company had followed the ACM code of ethics, they would have not had trouble with leadership since this framework focuses on professionalism and leadership. For such a large scale project leadership has to be established and stay coherent throughout the whole process otherwise mistakes are bound to happen. If it had not been for poor management, the robot most likely would not have been delivered to the client, meaning that nobody would have been injured by the faulty design. In addition, after the accident the company which supplied these robots would have been investigated. Because the department was threatened of closure they started rushing the production by any means necessary which lead to some employees stealing code from others which would lead to legal consequences for the manufacturer of the robot. This could have been avoided by using the 10 commandments, one of which states that no one under any circumstances should steal. Furthermore, the company that received these robots could have declined them if they knew what was going on. Hence, the data ethics framework would have probably saved the operator’s life because not only did the CEO allow for 99.44% purity of the software but the tests were also falsified. If the customer had known this they would not have used these unreliable and unsafe machines. This just shows how every problem can be viewed and/or overcome differently.
\section*{Conclusion}
To sum up, I believe the common good framework is a great framework but it depends on the field of work. Healthcare and politics benefit greatly by applying this mindset but companies should use different approaches for their work because sometimes we have to make sacrifices in order to advance further. Of course, it requires a great amount of discipline to obey the rules and now allow your gut feeling to take over. We are only human.
\end{document}
